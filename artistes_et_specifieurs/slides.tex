%\documentclass[notes=only]{beamer}
\documentclass{beamer}
\usepackage[french]{babel}
\usepackage[utf8]{inputenc}
\usetheme[progressbar=foot]{metropolis}
\usepackage{pgfpages}
%\setbeameroption{show notes on second screen}
\usepackage{soul}
\usepackage{fp}


\newcommand\twocolumns[3]{
  \FPeval{\leftsize}{#1}
  \FPeval{\rightsize}{1 - #1}

  \begin{columns}
    \begin{column}{\leftsize\textwidth}
      #2
    \end{column}
    \begin{column}{\rightsize\textwidth}
      #3
    \end{column}
  \end{columns}
}


\metroset{block=fill}

\newcommand{\singleimage}[2][\textwidth]{ \center \includegraphics[width=#1]{#2} }
\newcommand{\singletext}[1]{ \center{ \Large #1} }


%\setbeamertemplate{frame footer}{
%  \texttt{#vsmp}, \texttt{http://aidants-keilana.fr/}
%}


\title{Artistes et Spécifieurs}

\author{
  Virginie Albiez \texttt{<virginie.albiez@aidants-keilana.fr>} \\
  \\[5mm]
  \includegraphics[height=3cm]{includes/logo} \hfill
}


\begin{document}


\frame{\titlepage}


\begin{frame}{Qui}
  %\singleimage[0.6\textwidth]{includes/virginie_albiez}
  \begin{center}
    Virginie Albiez \\
    \texttt{$<$virginie.albiez@aidants-keilana.fr$>$} \\
  \end{center}
\end{frame}


\begin{frame}{Suivre nos pieds}
  \twocolumns{0.3}{
    \singleimage{includes/foots.png}
  }{
    \Large{Si vous n’apprenez rien ou que vous ne contribuez pas, suivez vos pieds !}
  }
\end{frame}


\begin{frame}{Objectifs de cette présentation}
  \begin{itemize}[<+->]
    \item Qu'êtes-vous venu chercher ?
    \item A quoi verrez-vous que c'était positif pour vous ?
    \item Qu'en pense votre voisin ou votre voisine ?
  \end{itemize}
\end{frame}


\begin{frame}{Mes objectifs de cette présentation}
  \begin{itemize}[<+->]
    \item Un nouveau concept pour se parler et s'améliorer
    \item Des outils
    \item Le plaisir du partage
  \end{itemize}
\end{frame}


\begin{frame}{Perfection game}

  \begin{block}{Je mets la note : }
    0 1 2 3 4 5 6 \alert{7} 8 9 10
  \end{block}
  \begin{block}{Parce que j'ai beaucoup aimé}
      \begin{itemize}
        \item
        \item
      \end{itemize}
  \end{block}
  \begin{block}{J'aurais mis 10 si}
      \begin{itemize}
        \item
        \item
      \end{itemize}
  \end{block}


    Des éléments concrets, actionnables, observables.
  \note{
    À quoi avez vous vue que c'était super.
    À quoi verriez vous que c'est encore mieux.

    Exemple du : "c'est de la merde" ou du "moins scolaire"
  }
\end{frame}


\section{Round 1}
\note{10 min}


\begin{frame}{Titre}
\end{frame}

\end{document}
